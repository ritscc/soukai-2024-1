\subsection*{2024年度秋学期全体総括}

\writtenBy{\president}{尾﨑}{真央}

本会の目的である「情報科学の研究,及びその成果の発表を活動の基本に会員相互の親睦を図り,
学術文化の創造と発展に寄与する」ことを達成するため,方針として以下の五つを立てた.
これらについてそれぞれ評価を行うことで2024年度春学期の総括とする.

\begin{itemize}
    \item 親睦を深める
    \item 規律ある行動
    \item 自己発信力の向上
    \item 会員間の技術向上
    \item 外部への情報発信
    \item 持続可能な運営
  \end{itemize}


\subsubsection*{親睦を深める}
例年とは異なり,プロジェクト活動は行わなかった.
しかし,その代わりに,勉強会やLT会を学期期間中に開催しつつ,夏休みにハッカソンを行うことで親睦を深める機会を設けた.

それ以外にも,新入生歓迎会を行い, 新入生と直接交流する機会を設けた.

春学期は,例年と比べるとイベントを通じて,親睦を深める機会が多かったと考える,

\subsubsection*{規律ある行動}
定例会議における遅刻欠席連絡を行うことはできたと考えられる,

また,BKCのサークルルームの利用についても綺麗に扱うことができたと考えられる,
加えて,OICで今年から利用可能となったサークルルームについても,綺麗に扱い,使ったモニターを元に戻すなどの規律のある使用ができたと考えられる,

\subsubsection*{自己発信力の向上}
定例会議におけるLT発表を通じて\secondGrade{}と\thirdGrade{}は自己発信力を向上させることができたと考えられる,

加えて,例年に比べるとハンズオン勉強会を多く開催し,勉強会において資料作りや記事作成などを行うことで,自己発信力を向上させる機会を増やすことができたと考えられる,

\subsubsection*{会員間の技術向上}
\secondGrade{}と\thirdGrade{}はLTと勉強会を通じて技術力向上を図ることができたと考えられる.

LTの内容は,簡単なものが多く,特に\firstGrade{}向けに技術が面白いと感じる内容をピックアップしていた方が多かったと考えられる,

また,ハッカソンを通して,\firstGrade{}には技術力向上の機会を提供することができたと考えられる,

\subsubsection*{外部への情報発信}
情報理工学部がOICに移転したことに伴い,サーバーを落としているため,Webページの公開ができていなかった,

Xでは,LTや勉強会に関する開催情報を発信していたため,情報発信は十分に行えていたと考えられる,

\subsubsection*{持続可能な運営}

まず,活動拠点について述べる,
2024年度から情報理工学部がOICに移転したことに伴い,活動拠点がOICとBKCとなった,
加えて,OICにもBKCにも新入会員が増えたため,全会員が集まり難くなってしまった,
そのため,全会員が集まりやすい,機会を作る機会を作らなければならないと考えられる,

次に,Webサイトについて述べる,
OIC移転に伴い,サーバーを落としたため,Webページの公開ができていなかった.
ハッカソンの開催に伴って,Webサイトで活動報告を行えなかったことや,Xでアドカレの情報を欲している方もいたため,Webサイトの公開ができていなかったことは反省点である.
クラウド移行も検討しているが,本会のドメイン問題などあるため,西村先生と協力して解決していくことが必要となる.

最後に,イベントについて述べる,
今回,Monkey Hack Partyというハッカソンを開催したが,他団体との繋がりと企業との繋がりができた良いイベントになったと考えている.
団体としては,Rig,Rist,Ri-one,Ri-one,といった団体に所属している学生が参加し,横のつながりを作れたと考える.
企業としては,ゆめみやGMOやサポーターズといった企業に協力をお願いし,参加学生に縦のつながりも作ることができたと考えられる.
このようなイベントがとても本会にとって刺激のあるイベントになったと考えられるため,今後も勉強会などで企業もよびつつ大きく開催することも大切だと考える.


