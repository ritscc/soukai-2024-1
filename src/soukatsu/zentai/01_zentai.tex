\subsection*{2024年度秋学期全体総括}

\writtenBy{\president}{尾﨑}{真央}

本会の目的である「情報科学の研究,及びその成果の発表を活動の基本に会員相互の親睦を図り,
学術文化の創造と発展に寄与する」ことを達成するため,方針として以下の五つを立てた.
これらについてそれぞれ評価を行うことで2024年度春学期の総括とする.

\begin{itemize}
    \item 親睦を深める
    \item 規律ある行動
    \item 自己発信力の向上
    \item 会員間の技術向上
    \item 外部への情報発信
    \item 持続可能な運営
  \end{itemize}


\subsubsection*{親睦を深める}
例年とは異なり,プロジェクト活動は行わなかった.
しかし、その代わりに、勉強会やLT会を学期期間中に開催しつつ、夏休みにハッカソンを行うことで親睦を深める機会を設けた.

それ以外にも,新入生歓迎会を行い, 新入生と直接交流する機会を設けた.

春学期は、例年と比べるとイベントを通じて、親睦を深める機会が多かったと考える。

\subsubsection*{規律ある行動}
例会における遅刻欠席連絡を行うことはできたと考えられる。

また、BKCの部室の利用についても綺麗に扱うことができたと考えられる。
加えて、OICで今年から利用可能となった部室についても、綺麗に扱い、使ったモニターを元に戻す等の規律のある使用ができたと考えられる。

\subsubsection*{自己発信力の向上}
例会におけるLT発表を通じて\secondGrade{}と\thirdGrade{}は自己発信力を向上させることができたと考えられる。

加えて、例年に比べるとハンズオン勉強会を多く開催し、勉強会において資料作りや記事作成等を行うことで、自己発信力を向上させる機会を増やすことができたと考えられる。