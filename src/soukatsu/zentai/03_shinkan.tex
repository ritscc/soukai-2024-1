\subsection*{新歓総括}

%\writtenBy{\president}{長張}{快}
%\writtenBy{\subPresident}{長張}{快}
%\writtenBy{\firstGrade}{長張}{快}
%\writtenBy{\secondGrade}{長張}{快}
\writtenBy{\thirdGrade}{長張}{快}
%\writtenBy{\fourthGrade}{長張}{快}

2023年度春学期の新歓の目的は,以下の2点であった.

\begin{itemize}
    \item 新入生に会の活動内容や活動方針について知ってもらう
    \item 新入生に会に興味を持ってもらう
\end{itemize}

これらの目的を達成するため,以下の4点の目標を掲げた.

\begin{itemize}
    \item LT会に参加してもらう
    \item 気軽にサークルルームに来てもらう
    \item 新入生に本会でやりたい事を見つけてもらう
    \item 新入生の中長期的な定着
\end{itemize}

これらの目標を達成するため,以下の五つの手法をとった
\begin{itemize}
    \item 大学側が提示してくるイベントに積極的に参加する
    \item SNSの利用(X,Discord,LINE)
    \item 新しいビラ・紹介スライドの作成
    \item オンラインでの新歓活動
    \item 定期的な勉強会(ハンズオン多め)
\end{itemize}

対面ブースの他に大学主催の新歓団体企画に応募,開催した.
対面ブースでは新しいビラ配布は150枚ほど配ることができた.
新歓団体企画では,VR体験に興味を持って来てくれる人が多かったと思われる.
DTMもできるということもかなり興味を引くポイントであった.
SNSの利用では,新歓用のDiscord鯖,LINEのオープンチャットを作成した.
X,Discord鯖,LINEのオープンチャットでの告知が功を奏し,多くの新入生に参加してもらえた.
定期的な勉強会は,新入生の中長期的な定着を達成するための手法である.
通年,Welcomeゼミ終了後すぐにLT会やプロジェクト活動といった通常活動に戻っていたが,多くの新入生がついていけないという問題があった.
そこで,2024年度は春学期にプロジェクト活動とWelcomeゼミを開催せず,毎週の勉強会で,新入生のプログラミングへの関心,基礎開発能力,サークル間でのコミュニケーション向上に勤めた.

以上の五つの手法により,2024年度における新歓の目的は達成されたと考えられる.
