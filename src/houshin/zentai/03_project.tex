\subsection*{プロジェクト活動方針}

\writtenBy{\thirdGrade}{尾崎}{真央}

\subsubsection*{目的}
プロジェクト活動の目的は,情報科学の研究をし,その成果の発表を活動の基本として会員相互の親睦を図るとともに学術文化の創造と発展に寄与する.

\subsection*{目標}
個人のみならずグループ活動としての経験を得る.
活動を通して技術力の向上を図る.
活動によって得られた成果を外部公開する.

\subsubsection*{活動内容}
プロジェクト内容は,コンピューターを使用した学習を通したものに限定する.

\subsubsection*{設立}
設立条件は以下のようにする.

一つは,プロジェクトのメンバーは三人以上とする.
二つは,リーダーのを選定し,そのリーダーは複数プロジェクトでの兼任禁止とする.

以上の条件を満たしたプロジェクトは設立とする.

\subsubsection*{メンバー募集}
定例会議でリーダーがそれぞれプロジェクトの説明を行い,その次の定例会議で締め切る.

募集はGoogleフォームを用いて行う.

\subsubsection*{プロジェクト解散}
プロジェクトに配属後,班員が3人未満もしくはリーダーが欠けた場合,そのプロジェクトは趣意書をもって理由を記述したのちに,上回生会議によって解散される.

\subsubsection*{プロジェクト発表会}
プロジェクトで得た知見や技術を共有する場として成果発表を行う.

発表形式は以下のようにする.
一つ目は,事前資料を何かしらで準備する.
二つ目は,形式はスライド発表にし,質疑応答の時間を設ける.
三つ目は,対面orオンラインの形式は状況によって変更する.

\subsubsection*{プロジェクトの運営}
秋学期のプロジェクト運営では,週報ではなく中間発表を設ける.
12月にどの程度進んでいるか定例会議で発表し,プロジェクトの運営状況を把握する.
